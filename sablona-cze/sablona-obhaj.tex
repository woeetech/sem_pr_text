% Soubory musí být v kódování, které je nastaveno v příkazu \usepackage[...]{inputenc}

\documentclass[%        Základní nastavení
  %draft,    				  % Testovací překlad
  12pt,       				% Velikost základního písma je 12 bodů
	t,                  % obsah slajdů bude vždy začínat od shora (nebude vertikálně centrovaný)
	aspectratio=1610,   % poměr stran bude 16:10 (všechny projektory v učebnách na Technické 12 Brno),
	                    % další volby jsou 43, 149, 169, 54, 32.
	unicode,						% Záložky a informace budou v kódování unicode
]{beamer}				    	% Dokument třídy 'zpráva', vhodná pro sazbu závěrečných prací s kapitolami
%\usepackage{etex}

\usepackage[utf8]		  % Kódování zdrojových souborů je v UTF-8
	{inputenc}					% Balíček pro nastavení kódování zdrojových souborů
	
\usepackage{graphicx} % Balíček 'graphicx' pro vkládání obrázků
											% Nutné pro vložení logotypů školy a fakulty

\usepackage[          % Balíček 'acronym' pro sazby zkratek a symbolů
	nohyperlinks				% Nebudou tvořeny hypertextové odkazy do seznamu zkratek
]{acronym}						
											% Nutné pro použití prostředí 'acronym' balíčku 'thesis'

%% Balíček hyperref je volán třídou beamer automaticky, proto není třeba následujícího kódu:
%\usepackage[
%	breaklinks=true,		% Hypertextové odkazy mohou obsahovat zalomení řádku
%	hypertexnames=false % Názvy hypertextových odkazů budou tvořeny
%											% nezávisle na názvech TeXu
%]{hyperref}						% Balíček 'hyperref' pro sazbu hypertextových odkazů
%											% Nutné pro použití příkazu 'nastavenipdf' balíčku 'thesis'

\usepackage{cmap} 		% Balíček cmap zajišťuje, že PDF vytvořené `pdflatexem' je
											% plně "prohledávatelné" a "kopírovatelné"

%\usepackage{upgreek}	% Balíček pro sazbu stojatých řeckých písmem
											%% např. stojaté pí: \uppi
											%% např. stojaté mí: \upmu (použitelné třeba v mikrometrech)
											%% pozor, grafická nekompatibilita s fonty typu Computer Modern!

%\usepackage{amsmath} %balíček pro sabu náročnější matematiky

\usepackage{booktabs} % Balíček, který umožňuje v tabulce používat
                      % příkazy \toprule, \midrule, \bottomrule


%%%%%%%%%%%%%%%%%%%%%%%%%%%%%%%%%%%%%%%%%%%%%%%%%%%%%%%%%%%%%%%%%
%%%%%%      Definice informací o dokumentu             %%%%%%%%%%
%%%%%%%%%%%%%%%%%%%%%%%%%%%%%%%%%%%%%%%%%%%%%%%%%%%%%%%%%%%%%%%%%

\input{nastaveni}      % v tomto souboru doplňte údaje o sobě, o názvu práce...
                       % (tento soubor je sdílený s textem práce)

%%%%%%%%%%%%%%%%%%%%%%%%%%%%%%%%%%%%%%%%%%%%%%%%%%%%%%%%%%%%%%%%%%%%%%%%

%%%%%%%%%%%%%%%%%%%%%%%%%%%%%%%%%%%%%%%%%%%%%%%%%%%%%%%%%%%%%%%%%%%%%%%%
%%%%%%     Nastavení polí ve Vlastnostech dokumentu PDF      %%%%%%%%%%%
%%%%%%%%%%%%%%%%%%%%%%%%%%%%%%%%%%%%%%%%%%%%%%%%%%%%%%%%%%%%%%%%%%%%%%%%
%% Při vloženém balíčku 'hyperref' lze použít příkaz '\pdfsettings'
\pdfsettings
%  Nastavení polí je možné provést také ručně příkazem:
%\hypersetup{
%  pdftitle={Název studentské práce},    	% Pole 'Document Title'
%  pdfauthor={Autor studenstké práce},   	% Pole 'Author'
%  pdfsubject={Typ práce}, 						  	% Pole 'Subject'
%  pdfkeywords={Klíčová slova}           	% Pole 'Keywords'
%}
\hypersetup{pdfpagemode=FullScreen}       % otevření rovnou v režimu celé obrazovky
%%%%%%%%%%%%%%%%%%%%%%%%%%%%%%%%%%%%%%%%%%%%%%%%%%%%%%%%%%%%%%%%%%%%%%%

\usetheme{VUT} 				% barvy a rozložení prezentace odpovídající VUT FEKT
% alternativně lze použít jiná berevná témata, ale bez záruky. Například: 
%\usetheme{Darmstadt} \usecolortheme{default2}
\logoheader					% vytvoření zkráceného loga VUT FEKT v hlavičce slajdu, nechte odkomentované

\usepackage[formats]{listings}


\lstset{              % nastavení
%	Definice jazyka použitého ve výpisech
%    language=[LaTeX]{TeX},	% LaTeX
%	language={Matlab},		% Matlab
	language={C},           % jazyk C
	basicstyle=\ttfamily,	% definice základního stylu písma
	tabsize=2,			% definice velikosti tabulátoru
	inputencoding=utf8,         % pro soubory uložené v kódování UTF-8
		columns=fixed,  %fixed nebo flexible,
		fontadjust=true %licovani sloupcu
	extendedchars=true,
	literate=%  definice symbolů s diakritikou
	{á}{{\'a}}1
	{č}{{\v{c}}}1
	{ď}{{\v{d}}}1
	{é}{{\'e}}1
	{ě}{{\v{e}}}1
	{í}{{\'i}}1
	{ň}{{\v{n}}}1
	{ó}{{\'o}}1
	{ř}{{\v{r}}}1
	{š}{{\v{s}}}1
	{ť}{{\v{t}}}1
	{ú}{{\'u}}1
	{ů}{{\r{u}}}1
	{ý}{{\'y}}1
	{ž}{{\v{z}}}1
	{Á}{{\'A}}1
	{Č}{{\v{C}}}1
	{Ď}{{\v{D}}}1
	{É}{{\'E}}1
	{Ě}{{\v{E}}}1
	{Í}{{\'I}}1
	{Ň}{{\v{N}}}1
	{Ó}{{\'O}}1
	{Ř}{{\v{R}}}1
	{Š}{{\v{S}}}1
	{Ť}{{\v{T}}}1
	{Ú}{{\'U}}1
	{Ů}{{\r{U}}}1
	{Ý}{{\'Y}}1
	{Ž}{{\v{Z}}}1
}
	
\lstset{
    frame=single,
    keywordstyle=\color{blue},
    commentstyle=\color{olive},
    basicstyle=\ttfamily\small,
    emph={byte, bool},
    emphstyle=\color{blue}
	}
	

\begin{document}

% v případě zakomentování následujícího se zobrazí v pravém dolním rohu slajdů klikatelné navigační symboly 
\disablenavigationsymbols

% titulní snímek, vysazen bez horních, dolních a postranních lišt (volba plain),
% není tak vysazen ani nadpis snímku
\maketitle

\newcommand{\bytes}[4]{
    \begin{center}
        \large{\texttt{0x#1, 0x#2, 0x#3, 0x#4}}
    \end{center}
}

%%%%%%%%%%%%%%%%%%%%%%%%%%%%%%%%%%%%%%%%%%%%%%%%%%%%%%%%%%%%%%%%%%%%%%%
% 1. snímek s cíli (zadaním) práce
\begin{frame} 
	% nadpis snímku
	\frametitle{Cíle práce}
	\begin{enumerate}
		\item Navrhnout a zkonstruovat testovací přípravek (HW i SW)
		\item Navrhnout komunikační protokol
		\item Naprogramovat síťový editor pro pohodlnou správu všech přípravků na síti
	\end{enumerate}
\end{frame}

\begin{frame}
	\frametitle{Testovací přípravek -- HW}
	\begin{itemize}
		\item Platforma \textbf{Arduino UNO} + \textbf{WizNet Shield}
		\item Vyvinuto na nepájivém kontaktním poli
		\item Využito standardních vstupních a výstupních obvodů v souladu s MIDI normou
	\end{itemize}
	\begin{figure}
		\centering
		\includegraphics[height=0.4\textheight ]{obrazky/Pripravek.jpg}
		\caption{Fotografie přípravku}
	\end{figure}
\end{frame}

\begin{frame}[fragile]
	\frametitle{Testovací přípravek -- SW}
	\begin{itemize}
		\item Program v jazyce \texttt{C/C++}
		\item Využito knihovny \texttt{EthernetUdp.h} rozšířené pro účely projektu
		\item Pokročilé možnosti směrování docíleny implementací \textbf{databáze spojení}
	\end{itemize}
	\begin{lstlisting}
typedef struct subscription
{
	byte srcdstChannel;
	byte dstIPnib;
}
subscription subscriptions[MAX_SUBS];
	\end{lstlisting}
\end{frame}

\begin{frame}
	\frametitle{MoE Protokol}
	\begin{itemize}
		\item Mezi zařízeními jsou posílány 4-bajtové zprávy
		\item První bajt~-- \textbf{MoE značka}~-- definuje účel zprávy
	\end{itemize}
	\vspace{3ex}
	\begin{block}{Ukázkové zprávy}
		\bytes{A3}{90}{45}{7F}
		\bytes{08}{08}{08}{08}
	\end{block}
	\vspace{3ex}
	\begin{itemize}
		\item Zprávy jsou odesílány pomocí \textbf{UDP protokolu} v síti s maskou \textbf{255.255.255.0} na portu \textbf{50\,000}
	\end{itemize}
\end{frame}

\begin{frame}
	\frametitle{Síťový editor}
	\begin{columns}[T]
		\begin{column}{0.4\textwidth}
			\begin{itemize}
				\item Slouží ke správě všech MoE zařízení v~lokální síti
				\item Běží na PC/Mac, který je k~síti připojen fyzickým rozhraním nebo pomocí Wi-Fi
				\item Příkazy
				\begin{itemize}
					\item \texttt{print}
					\item \texttt{add}
					\item \texttt{del}
					\item \texttt{reload}
				\end{itemize}
				\item \uv{Makra} pro hromadné přidání/smazání
			\end{itemize}
		\end{column}
		\begin{column}{0.6\textwidth}
			\begin{figure}
				\includegraphics[width=\linewidth]{obrazky/MoE_Editor_2.png}
				\caption{Konzolová aplikace \textbf{MoE Matrix Editor}}
			\end{figure}
		\end{column}
	\end{columns}
\end{frame}

\begin{frame}
	\frametitle{Výsledky \& prostor pro optimalizaci}
	\begin{columns}[T]
		\begin{column}{.6\textwidth}
			\begin{figure}
				\includegraphics[width=\linewidth]{obrazky/Mereni_1kanal.png}
				\caption{Měření latence mezi dvěma přípravky}
			\end{figure}
		\end{column}
		\begin{column}{.4\textwidth}
			\begin{itemize}
				\item Prototyp je funkční, k síti se připojí buď s napevno nastavenou IP adresou, nebo s pomocí DHCP serveru
				\item Síťový editor také funguje
				\item Zatím pouze \textbf{tříbajtové MIDI}
				\item Latence \textbf{2,18--20,92\,ms}
				\item Příjem maximálně dvou MIDI příkazů za sebou
			\end{itemize}
		\end{column}
	\end{columns}
	
\end{frame}
%%%%%%%%%%%%%

%%%%%%%%%%%%%




%%%%%%%%%%%%%


%%%%%%%%%%%%%


% podekovani
\begin{frame}[c] 
% bez nadpisu snímku
	\frametitle{\mbox{ }}
	\begin{center}
		{\Huge Děkuji za pozornost!}
	\end{center}
\end{frame}

% otázky oponenta


\end{document}

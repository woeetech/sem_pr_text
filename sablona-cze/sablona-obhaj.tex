% Soubory musí být v kódování, které je nastaveno v příkazu \usepackage[...]{inputenc}

\documentclass[%        Základní nastavení
  %draft,    				  % Testovací překlad
  12pt,       				% Velikost základního písma je 12 bodů
	t,                  % obsah slajdů bude vždy začínat od shora (nebude vertikálně centrovaný)
	aspectratio=1610,   % poměr stran bude 16:10 (všechny projektory v učebnách na Technické 12 Brno),
	                    % další volby jsou 43, 149, 169, 54, 32.
	unicode,						% Záložky a informace budou v kódování unicode
]{beamer}				    	% Dokument třídy 'zpráva', vhodná pro sazbu závěrečných prací s kapitolami
%\usepackage{etex}

\usepackage[utf8]		  % Kódování zdrojových souborů je v UTF-8
	{inputenc}					% Balíček pro nastavení kódování zdrojových souborů
	
\usepackage{graphicx} % Balíček 'graphicx' pro vkládání obrázků
											% Nutné pro vložení logotypů školy a fakulty

\usepackage[          % Balíček 'acronym' pro sazby zkratek a symbolů
	nohyperlinks				% Nebudou tvořeny hypertextové odkazy do seznamu zkratek
]{acronym}						
											% Nutné pro použití prostředí 'acronym' balíčku 'thesis'

%% Balíček hyperref je volán třídou beamer automaticky, proto není třeba následujícího kódu:
%\usepackage[
%	breaklinks=true,		% Hypertextové odkazy mohou obsahovat zalomení řádku
%	hypertexnames=false % Názvy hypertextových odkazů budou tvořeny
%											% nezávisle na názvech TeXu
%]{hyperref}						% Balíček 'hyperref' pro sazbu hypertextových odkazů
%											% Nutné pro použití příkazu 'nastavenipdf' balíčku 'thesis'

\usepackage{cmap} 		% Balíček cmap zajišťuje, že PDF vytvořené `pdflatexem' je
											% plně "prohledávatelné" a "kopírovatelné"

%\usepackage{upgreek}	% Balíček pro sazbu stojatých řeckých písmem
											%% např. stojaté pí: \uppi
											%% např. stojaté mí: \upmu (použitelné třeba v mikrometrech)
											%% pozor, grafická nekompatibilita s fonty typu Computer Modern!

%\usepackage{amsmath} %balíček pro sabu náročnější matematiky

\usepackage{booktabs} % Balíček, který umožňuje v tabulce používat
                      % příkazy \toprule, \midrule, \bottomrule


%%%%%%%%%%%%%%%%%%%%%%%%%%%%%%%%%%%%%%%%%%%%%%%%%%%%%%%%%%%%%%%%%
%%%%%%      Definice informací o dokumentu             %%%%%%%%%%
%%%%%%%%%%%%%%%%%%%%%%%%%%%%%%%%%%%%%%%%%%%%%%%%%%%%%%%%%%%%%%%%%

\input{nastaveni}      % v tomto souboru doplňte údaje o sobě, o názvu práce...
                       % (tento soubor je sdílený s textem práce)

%%%%%%%%%%%%%%%%%%%%%%%%%%%%%%%%%%%%%%%%%%%%%%%%%%%%%%%%%%%%%%%%%%%%%%%%

%%%%%%%%%%%%%%%%%%%%%%%%%%%%%%%%%%%%%%%%%%%%%%%%%%%%%%%%%%%%%%%%%%%%%%%%
%%%%%%     Nastavení polí ve Vlastnostech dokumentu PDF      %%%%%%%%%%%
%%%%%%%%%%%%%%%%%%%%%%%%%%%%%%%%%%%%%%%%%%%%%%%%%%%%%%%%%%%%%%%%%%%%%%%%
%% Při vloženém balíčku 'hyperref' lze použít příkaz '\pdfsettings'
\pdfsettings
%  Nastavení polí je možné provést také ručně příkazem:
%\hypersetup{
%  pdftitle={Název studentské práce},    	% Pole 'Document Title'
%  pdfauthor={Autor studenstké práce},   	% Pole 'Author'
%  pdfsubject={Typ práce}, 						  	% Pole 'Subject'
%  pdfkeywords={Klíčová slova}           	% Pole 'Keywords'
%}
\hypersetup{pdfpagemode=FullScreen}       % otevření rovnou v režimu celé obrazovky
%%%%%%%%%%%%%%%%%%%%%%%%%%%%%%%%%%%%%%%%%%%%%%%%%%%%%%%%%%%%%%%%%%%%%%%

\usetheme{VUT} 				% barvy a rozložení prezentace odpovídající VUT FEKT
% alternativně lze použít jiná berevná témata, ale bez záruky. Například: 
%\usetheme{Darmstadt} \usecolortheme{default2}
\logoheader					% vytvoření zkráceného loga VUT FEKT v hlavičce slajdu, nechte odkomentované

\usepackage[formats]{listings}
\lstset{
    frame=single,
    keywordstyle=\color{blue},
    commentstyle=\color{olive},
    basicstyle=\ttfamily\small,
    emph={byte, bool},
    emphstyle=\color{blue}
    }

\begin{document}

% v případě zakomentování následujícího se zobrazí v pravém dolním rohu slajdů klikatelné navigační symboly 
\disablenavigationsymbols

% titulní snímek, vysazen bez horních, dolních a postranních lišt (volba plain),
% není tak vysazen ani nadpis snímku
\maketitle

\newcommand{\bytes}[4]{
    \begin{center}
        \large{\texttt{0x#1, 0x#2, 0x#3, 0x#4}}
    \end{center}
}

%%%%%%%%%%%%%%%%%%%%%%%%%%%%%%%%%%%%%%%%%%%%%%%%%%%%%%%%%%%%%%%%%%%%%%%
% 1. snímek s cíli (zadaním) práce
\begin{frame} 
	% nadpis snímku
	\frametitle{Cíle práce}
	\begin{itemize}
		\item Navrhnout a zkonstruovat testovací přípravek (HW i SW)
		\item Navrhnout komunikační protokol
		\item Naprogramovat síťový editor pro pohodlnou správu všech přípravků na síti
	\end{itemize}
\end{frame}

\begin{frame}
	\frametitle{Testovací přípravek -- HW}
	\begin{itemize}
		\item Platforma \textbf{Arduino UNO} + \textbf{WizNet Shield}
		\item Vyvinuto na nepájivém kontaktním poli
		\item Využito standardních vstupních a výstupních obvodů v souladu s MIDI normou
	\end{itemize}
	\begin{figure}
		\centering
		\includegraphics[height=0.4\textheight ]{obrazky/Pripravek.jpg}
		\caption{Fotografie přípravku}
	\end{figure}
\end{frame}

\begin{frame}
	\frametitle{Testovací přípravek -- SW}
	\begin{itemize}
		\item Program v jazyce \texttt{C/C++}
		\item Využito knihovny \texttt{EthernetUdp.h} rozšířené pro účely projektu
		\item Rozšířené možnosti směrování docíleny implementací \textbf{databáze spojení}
		\item 
	\end{itemize}
\end{frame}

\begin{frame}
	\frametitle{MoE Protokol}
	\begin{itemize}
		\item Mezi zařízeními jsou vyměňovány 4-bajtové zprávy
		\item První bajt~--\textbf{MoE značka} definuje účel zprávy
	\end{itemize}
	\vspace{3ex}
	\begin{block}{Ukázkové zprávy}
		\bytes{A3}{90}{45}{7F}
		\bytes{08}{08}{08}{08}
	\end{block}
	\vspace{3ex}
	\begin{itemize}
		\item Zprávy jsou odesílány pomocí \textbf{UDP protokolu} na síti s maskou \textbf{255.255.255.0} na portu \textbf{50\,000}
	\end{itemize}
\end{frame}

\begin{frame}
	\frametitle{Síťový editor}
	\begin{columns}[T]
		\begin{column}{0.4\textwidth}
			\begin{itemize}
				\item Slouží ke správě všech MoE zařízení v~lokální síti
				\item Běží na PC/Mac, který je k~síti připojen fyzickým rozhraním nebo pomocí Wi-Fi
				\item Příkazy
				\begin{itemize}
					\item \texttt{print}
					\item \texttt{add}
					\item \texttt{del}
					\item \texttt{reload}
				\end{itemize}
				\item \uv{Makra} pro hromadné přidání/smazání
			\end{itemize}
		\end{column}
		\begin{column}{0.6\textwidth}
			\begin{figure}
				\includegraphics[width=\linewidth]{obrazky/MoE_Editor_2.png}
				\caption{Konzolová aplikace \textbf{MoE Matrix Editor}}
			\end{figure}
		\end{column}
	\end{columns}
\end{frame}

\begin{frame}
	\frametitle{Výsledky \& prostor pro optimalizaci}
	\begin{itemize}
		\item Prototyp je funkční, k síti se připojí buď s napevno nastavenou IP adresou, nebo s pomocí DHCP serveru
		\item 
	\end{itemize}
\end{frame}
%%%%%%%%%%%%%
\begin{frame} 
	\frametitle{Klíčové nástroje}

	% prostředí 'alertblock', které slouží pro zdůraznění informace
	\begin{alertblock}{Pro práci je klíčový Eulerův vzorec}
		$$\eul^{\jmag x}=\cos x + \jmag\sin x$$
	\end{alertblock}

	\vspace{4ex}
	Eulerova identita je speciálním případem tohoto vzorce, jestliže dosadíme $x=\uppi$\,:

	% prostředí 'block', které slouží jako informativní
	\begin{block}{Eulerova identita}
		$$\eul^{\jmag \uppi}=\cos \uppi + \jmag\sin \uppi,$$\\
		odkud vyplývá
		$$\eul^{\jmag \uppi}+1=0.$$
	\end{block}
\end{frame} 


%%%%%%%%%%%%%
\begin{frame} 
	\frametitle{Plošný spoj}
	
	\begin{columns}[T] 								% prostředí sloupce s umístěním nahoře
		\begin{column}{0.4\textwidth}		% první sloupec
			Obrázek znázorňuje model:\\[2ex]
			%
			\begin{itemize}
				\item Deska
				\item Součástky
				\item Signály
				\item Napájení
			\end{itemize}
		\end{column}
		%
		\begin{column}{0.6\textwidth}		% druhý sloupec
			\begin{figure}%	
				\centering
				\vspace{1cm}	              % horizontální mezera
				\includegraphics[width=0.8\columnwidth]{obrazky/MoE_editor_1.png}
				%lze vložit popisek, ale povetšinou je to v prezentaci zbytečné
				%\caption{Popisek obrázku}%
				%\label{obr:ukazka}
			\end{figure}
		\end{column}
	\end{columns}											% ukončení prostředí sloupce
\end{frame}


%%%%%%%%%%%%%
\begin{frame} 
	\frametitle{Výsledky}
	\vspace{1cm}
	\begin{table}[]
		\centering
		\caption{Výsledky měření mobilních sítí}
		\label{tab:tabulka}
			\begin{tabular}{lcc}
				\toprule
					Technologie  & Rychlost stahování [kB/s] & Rychlost nahrávání [kB/s] \\
				\midrule
					GPRS (2,5G)	& 7,2 	& 3,6\\
					UMTS 3G     & 48 		& 48\\
					HSPA (3,5G)	&	1\,706	&	720\\
					LTE (4G) 		& 40\,750 & 10\,750\\
				\bottomrule                                       
			\end{tabular}
	\end{table}
\end{frame}


%%%%%%%%%%%%%
\begin{frame} 
	\frametitle{Závěr}
	\dots
\end{frame}


% podekovani
\begin{frame}[c] 
% bez nadpisu snímku
	\frametitle{\mbox{ }}
	\begin{center}
		{\Huge Děkuji za pozornost!}
	\end{center}
\end{frame}

% otázky oponenta
\frame{
\frametitle{Otázky oponenta}
	\emph{Jaká je souvislost Vašeho vzorce (1.2) s~Maxwellovými rovnicemi v~integrálním tvaru?}\\[2ex]
	%
	Již staří Římané\,\dots
}

\end{document}

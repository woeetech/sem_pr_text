\begin{circuitikz}[]
    \draw (0,0) circle(0.5cm) ++ (0,-0.9) node {MIDI IN};
    \draw (0,0.3) circle(0.07cm) node(midi2){} ++(0,0.35) node {\tiny 2};
    \draw (-0.3,0) circle(0.07cm) node(midi3){} ++(-0.35,0) node {\tiny 3};
    \draw (0.3,0) circle(0.07cm) node(midi1){} ++(0.35,0) node {\tiny 1};
    \draw (0.22,0.22) circle(0.07cm) node(midi4){} ++(0.32,0.15) node {\tiny 4};
    \draw (-0.22, 0.22) circle(0.07cm) node(midi5){} ++(-0.32,0.15) node {\tiny 5};
    
    \draw (midi4) -- (0.7,0.7) -- (0.7,1) -- (2,1);
    \draw (midi5) -- (-0.7,0.7) -- (-0.7,2) -- (2,2);
    
    \draw[densely dashed] (1.5,1.5) ellipse (0.3 and 0.9);
    \draw (1.5,0.6) to[short, -*] ++(0,0) node[ground]{};
    
    \draw (2,2) -- (4,2);
    \ctikzset{resistors/scale=0.7}
    \draw (2,1) to[R=\footnotesize $220\,\Omega$] (4,1);
    
    \ctikzset{diodes/scale=0.5}
    \draw (4,2) to [short,-*] ++(0,0) to[D*] (4,1) to[short, -*] ++(0,0) node[below]{\tiny 1N914};
    
    \draw (4,2) -- (5,2);
    \draw (4,1) -- (5,1);
    
    \draw[thick] (5,0.75) rectangle (7,2.25);
    \draw (6,0.75) node[below]{\tiny 6N138};
    
    \draw[densely dashed] (5,2) -- (5.5,2) -- (5.5,1) -- (5,1);
    \draw (5.5,1.5) to[D*] ++(0,0);
    
    \draw[thick, ->] (5.8,1.5) -- (6.3,1.5);
    
    \draw[very thick] (6.5,2) -- (6.5,1);
    \draw (6.5,1.5) -- (6.75,2)  to[short, -*] ++(1.25,0) to[short,-o] ++(1,0) node[right]{TO UART};
    \draw (6.5,1.5) to[short, i=$ $] (6.75,1) -- ++(0.5,0) -- (7.25,0.6) node[ground]{};
    
    \draw (6,2.25) to[short,-o] ++(0,1) node[above]{\small 5\,V};
    \draw (6,2.75) to[short,*-] ++(0,0) to[R=\footnotesize $280\,\Omega$] ++(2,0) to[short, -o] (8,-0.5);
    \draw (8,-0.9) node{MIDI THRU};
    
    %
    %dokument z roku 2014 upravuje:
    %
    
    
    
    
    
\end{circuitikz}
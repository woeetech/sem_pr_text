\cleardoublepage
\chapter*{\listofabbrevname}
\phantomsection
\addcontentsline{toc}{chapter}{\listofabbrevname}

\begin{acronym}[KolikMista]
    \acro{DAW}
    {Digital Audio Workstation -- digitální pracovní stanice pro náběr a úpravu vícestopého záznamu.}
    \acro{LED}
    {Light-Emiting Diode -- dioda, která vyzařuje viditelné světlo.}
    \acro{MIDI}
    {Musical Instrument Digital Interface -- digitální rozhraní hudebního nástroje.}
    \acro{MoE}
    {\acs{MIDI} over Ethernet~-- \acs{MIDI} po Ethernetu}
    \acro{MSb}
    {Most Significant bit -- nejvýznamnější bit (většinou v bajtu).}
    \acro{MSC}
    {\acs{MIDI} Show Control -- subprotokol pro ovládání scénické techniky, zejména pomocí příkazů GO, STOP, popř. RESUME. \cite{MIDIspecs}}
    \acro{RF}
    {Radio Frequency}
    \acro{UART}
    {Universal Asynchronous Receiver/Transmitter -- univerzální asynchronní přijímač/vysílač.}
    \acro{UDP}
    {User Datagram Protocol~-- jednoduchý síťový protokol, který umožňuje výměnu zpráv \uv{host-to-host} \cite{UDPpaper}.}
    \acro{USB}
    {Universal Serial Bus -- univerzální sériová sběrnice pro komunikace s periferními zařízeními.}







%	\acro{zkTemp}		% název
%		[Šířka levého sloupce Seznamu symbolů, veličin a zkratek]	%							% zkratka
%		{je určena šířkou parametru prostředí \texttt{acronym} (viz %řádek~1 výpisu zdrojáku na~str.\,\pageref{lst:zkratky})}
%											% rozvinutí zkratky
%
%	\acro{zkDummy}
%		[KolikMista]
%		{pouze ukázka vyhrazeného místa}
%
%	\acro{DSP}		% název/zkratka
%		{číslicové zpracování signálů -- Digital Signal Processing}
%											% rozvinutí zkratky
%	%%% bsymfvz
%	\acro{symfvz}						% název
%		[\ensuremath{f_\textind{vz}}] % symbol
%		{vzorkovací kmitočet}					% popis
%	%%% esymfvz

\end{acronym}

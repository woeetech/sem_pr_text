\chapter*{Závěr}
\phantomsection
\addcontentsline{toc}{chapter}{Závěr}

Výstupem této semestrální práce je funkční přípravek, který disponuje ethernetovým a vstupním i výstupním \acs{MIDI} rozhraním. Přípravek se automaticky připojí do počítačové sítě a dále funguje již autonomně. Dokáže přijmout \acs{MIDI} zprávu, přeformátovat ji v~souladu s~daným záznamem databáze spojení a s~využitím \acs{UDP} protokolu poslat správnému příjemci. Databáze spojení je taktéž editovatelná na dálku pomocí programu \emph{MoE Matrix Editor}. 

Stále však existuje mnoho prostoru pro optimalizaci. Pokud přípravek obdrží více než dvě \acs{MIDI} zprávy těsně za sebou, přestane přijímat jakékoliv další bajty a jeho funkčnost je do restartu přerušena. Absence implementace ošetření výjimek (dle kapitoly \ref{chpt:MIDIexcs}) zase prozatím brání využití přípravků tam, kde je příjem zpráv v módu Running Status nebo SysEx standardem.

Jako uspokojivé lze však vnímat dosažené hodnoty latence mezi přijetím \acs{MIDI} zprávy na sériové sběrnici jednoho zařízení a odesláním též zprávy na sériovou sběrnici zařízení druhého. Při tomto měření probíhala komunikace mezi samotnými zařízeními pochopitelně prostřednictvím počítačové sítě. Dle přílohy \ref{fig:Latency1} vykazuje dolní hranice celkové latence \acs{MoE} řešení $2{,}18\,\unit{ms}$ v~případě jediného záznamu v~databázi spojení. S~každým dalším záznamem pak latence přirozeně roste. Maximální naměřená hodnota latence (mezi první přijatou a poslední odeslanou zprávou) pak dosahovala $20{,}92\,\unit{ms}$. Při živé hudební produkci s~využitím \acs{MIDI} klávesového nástroje by neměla celková latence přesáhnout 5\,ms. S~touto premisou lze prohlásit, že \acs{MoE} řešení by mohlo být za určitých podmínek použitelné i~pro časově náročnější podmínky.
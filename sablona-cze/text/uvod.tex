\phantomsection
\addcontentsline{toc}{chapter}{Úvod}
\chapter*{Úvod}

Protokol \acs{MIDI}\footnote{\acl{MIDI}} je již dlouhá léta zavedeným standardem nejen pro komunikaci mezi elektronickými hudebními nástroji, ale také pro řízení studiové nebo scénické techniky, časovou synchronizaci dvou a více zařízení a podobně. Cílem této práce je vyvinout hardwarový přípravek s~vlastním softwarem, který adaptuje tento protokol pro použití v~rámci počítačové sítě. Mimo elementární přesun \acs{MIDI} zpráv by měl poskytnout i~komplexní možnosti směřování jednotlivých vstupních a výstupních \acs{MIDI} kanálů.

První kapitola je ve stručnosti věnována \acs{MIDI} protokolu jako takovému. Ve druhé kapitole je nastíněna konstrukce přípravku a v~obrysech popsán jeho program. Třetí kapitola nabízí bližší pohled na průběh komunikace a ve čtvrté kapitole je popsána funkčnost síťového editoru, jehož hlavní devíza tkví ve schopnosti ovládat všechny přípravky v~síti z~jednoho místa~-- řídícího PC.

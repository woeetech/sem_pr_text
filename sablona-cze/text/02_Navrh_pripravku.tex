\chapter{Návrh přípravku}
V~rámci semestrální práce byl vytvořen prototyp zařízení na platformě Arduino~UNO, která disponuje čipem ATMEL ATmega 328P. Arduinu sekunduje \linebreak Ethernet~Shield~V1, který je osazen čipem WizNet W5100. 

Přípravek umožňuje výměnu \acs{MIDI} příkazů přes lokální síť, využívá první čtyři vrstvy ISO/OSI modelu. Zároveň poskytuje pokročilé možnosti směrování jednotlivých \acs{MIDI} kanálů.

\section{Schéma}\label{chpt:Schema}
\begin{figure}[h]
    \centering
    \begin{circuitikz}[]
    \draw (0,0) circle(0.5cm) ++ (0,-0.9) node {MIDI IN};
    \draw (0,0.3) circle(0.07cm) node(m1p2){} ++(0,0.35) node {\tiny 2};
    \draw (-0.3,0) circle(0.07cm) node(m1p3){} ++(-0.35,0) node {\tiny 3};
    \draw (0.3,0) circle(0.07cm) node(m1p1){} ++(0.35,0) node {\tiny 1};
    \draw (0.22,0.22) circle(0.07cm) node(m1p4){} ++(0.32,0.15) node {\tiny 4};
    \draw (-0.22, 0.22) circle(0.07cm) node(m1p5){} ++(-0.32,0.15) node {\tiny 5};
    
    \draw (9,0) circle(0.5cm) ++ (0,-0.9) node {MIDI OUT};
    \draw (9,0.3) circle(0.07cm) node(m2p2){} ++(0.1,0.35) node {\tiny 2};
    \draw (8.7,0) circle(0.07cm) node(m23){} ++(-0.35,0) node {\tiny 3};
    \draw (9.3,0) circle(0.07cm) node(m2p1){} ++(0.35,0) node {\tiny 1};
    \draw (9.22,0.22) circle(0.07cm) node(m2p4){} ++(0.32,0.15) node {\tiny 4};
    \draw (8.78, 0.22) circle(0.07cm) node(m2p5){} ++(-0.32,0.15) node {\tiny 5};
    
    \ctikzset{resistors/scale=0.7}
    \ctikzset{diodes/scale=0.5}
    
    \draw (m1p5) -- (-0.7,0.7) -- (-0.7,4) -- (3,4) to[short, *-] ++(0,0) to[D*] ++(down: 1) to[short, -*] ++(0,0); 
    \draw (m1p4) -- (0.7,0.7) -- (0.7,3) -- (1,3) to[R, l= \footnotesize $R_1$] ++(right: 2);

    \draw (5,3.5) node[dipchip, 
        num pins = 8](OptoIsol){\footnotesize 6N138};

    \draw (3,4) -| (OptoIsol.pin 2);
    \draw (3,3) -| (OptoIsol.pin 3);
    
    \draw (9.1, 3.4) node[dipchip,
        num pins = 10,
        external pins width = 0,
        fill=white,
        hide numbers,
        no topmark]{};
    \draw (9,3.5) node[dipchip,
        num pins = 10,
        external pins width = 0,
        fill=white,
        align=center,
        hide numbers, no topmark](Arduino){\footnotesize{Arduino}\\\footnotesize{WizNet}};
    
    \draw (Arduino.north) node[below]{\tiny 5\,V};
    \coordinate (h1) at (7.3,3.5);
    \draw (Arduino.north) -- ++(up: 0.5) coordinate (h2);
    \draw (h2) to[short, -*] (h2 -| h1) coordinate (middle_top);
    \draw (OptoIsol.pin 8) coordinate (h3) -- (h3 |- h2) -- (middle_top);
    \draw (OptoIsol.pin 6) coordinate (hp6) to[short,-*] (hp6 -| middle_top);
    \draw (middle_top) to[R, l=\footnotesize $R_2$] (middle_top |- hp6);
    \draw (Arduino.pin 5) node[right]{\tiny 4} coordinate (ArP6) -| (middle_top |- hp6);

    \coordinate (h9) at (6.6,3.5);
    \draw (Arduino.south) node[above]{\tiny GND} -- ++(down: 0.5) coordinate (ArPG) -- (ArPG -| OptoIsol.pin 5) -- (OptoIsol.pin 5);
    
    \draw (OptoIsol.pin 7) -- (OptoIsol.pin 7 -| h9) -- (h9 |- OptoIsol.pin 6) to[R,l=\footnotesize$R_3$,-*] (h9 |- ArPG); 

    \draw (m2p4) -- (9.7,0.7) -- ++(right: 0.7) coordinate (h10);
    \draw (h2) to[short,*-] (h2 -| h10) to[R, l=\footnotesize$R_4$] (h10);

    \draw (m2p2) -- ++(up: 0.8) coordinate (m2p2up) -- (m2p2up -| Arduino.south) to[short,-*] (Arduino.south |- ArPG);
    
    \draw (m2p5) -- (8.3,0.7) -- ++(left: 0.6) coordinate(h11) -- (h11 |- Arduino.pin 4) -- (Arduino.pin 4) node[right]{\tiny 5};
    %\draw (OptoIsol.pin 8) -- (h1 |- h2);
    
    

\end{circuitikz}



    \begin{tabular}{l c l c}
        \small
        $R_1$ & $220\,\mathrm{\Omega}$ & $R_2$ & $470\,\mathrm{\Omega}$ \\
        $R_3$ & $10\,\mathrm{k\Omega}$ & $R_4$ & $220\,\mathrm{\Omega}$
    \end{tabular}
    \caption{Schéma prototypu \cite{Indest}} 
    \label{fig:schPrototype}
\end{figure}

Z obr. \ref{fig:schPrototype} je patrné zapojení přípravku. Je využito standardní zapojení \acs{MIDI} vstupu a~výstupu v souladu se schématy na obr.~\ref{fig:schMIDIout} a~\ref{fig:schMIDIin}. Jak již bylo řečeno v kapitole~\ref{chpt:MIDI}, \acs{MIDI} využívá pro komunikaci sériovou linku. V tomto případě je vstupní linka zapojena k~Arduinu na \texttt{pinu~2} a výstupní k~\texttt{pinu~3}\footnote{Arduino umožňuje softwarové přepnutí \texttt{pinů} z režimu výstupu na režim vstup a naopak.}. 

Ve schématu na obr.~\ref{fig:schPrototype} je pro zjednodušení značka \uv{Arduino} chápána jako \uv{Arduino\,+\,Wiznet Shield}.

Přípravek je v~této fázi vývoje napájen skrz \acs{USB} sběrnici Arduina napětím 5\,\unit{V}. 

%...%

\section{Protokol}





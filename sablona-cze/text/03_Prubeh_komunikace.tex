\chapter{Průběh komunikace}
V~této kapitole bude popsán průběh připojení přípravku k síti, automatické rozpoznání a zacházení s přijatými \acs{MIDI} a \acs{UDP} zprávami, spolu s reakcemi na příkazy odeslané síťovým editorem.

\section{Připojení k lokální síti}
Každému přípravku byla do prvních šesti bajtů \texttt{EEPROM} paměti uložena unikátní MAC adresa. Při zapnutí a následné inicializaci je adresa načtena a přiřazena. Nejprve proběhne pokus o získání IP adresy pomocí DHCP serveru sítě. V případě neúspěchu použije přípravek napevno nastavenou IP adresu. 

Díky knihovně \texttt{EthernetUdp.h} jsou výše zmíněné operace otázkou pouze několika málo řádků:

\begin{lstlisting}
if (Ethernet.begin())
{
    //DHCP server zdárně přidělil zařízení IP adresu
    Serial.println(Ethernet.localIP());
}
else
{
    Ethernet.begin(_myMac, _userIP);
    //Přidělení IP adresy napevno
    Serial.println(Ethernet.localIP());
}
\end{lstlisting}






\section{Upozornění na vlastní přítomnost}
Ihned po zdárném připojení k lokální síti je vyslána zpráva  \texttt{beacon}, jejíž účel je upozornit všechna zařízení v lokální síti na vlastní přítomnost. Zpráva se skládá ze čtyř bajtů:
\bytes{FF}{FF}{FF}{FF}
a je poslána na broadcast adresu sítě.

\begin{lstlisting}
_broadcastIP = Ethernet.localIP();
_broadcastIP[3] = 255;
EthernetUdp eUDP;
eUDP.begin(MOE_PORT);
eUDP.beginPacket(_broadcastIP, MOE_PORT);
eUDP.write(_beacon, sizeof(_beacon));
eUDP.endPacket();
\end{lstlisting}

\section{Rozpoznání zařízení na síti}
V případě přijetí zprávy \texttt{beacon} je rozpoznána IP adresa odesílatele a vytvořen nový záznam do databáze spojení \texttt{sub\-scrip\-tions}. V této fázi projektu přípravek napevno přiřazuje kanál 1  lokální příchozí \acs{MIDI} sběrnice kanálu 1 výstupní \acs{MIDI} sběrnice cílového zařízení. Unikátnost této zprávy v rámci běhu programu má však velice negativní efekt: Zařízení, keré je do sítě připojeno jako poslední nepřijme žádnou zprávu \texttt{beacon}, do své databáze si tedy žádné spojení nepřidá. Z toho důvodu byl vyvinut síťový editor, který umožňuje přidávat a mazat záznamy do databází spojení všech přípravků kdykoliv za běhu programu. 
\begin{lstlisting}
switch(_incomingUDP[0])
{
    //...
    case 0xFF:
        addSubscription(0, eUDP.remoteIP()[3], 0);
    break;
    //...
}
\end{lstlisting}

\section{Přijetí \acs{MIDI} zprávy}\label{chpt:PrijMIDI}
Při přijímání \acs{MIDI} zprávy je kontrolován vstupní buffer sériové sběrnice \texttt{pinu 5}. Pokud jsou v bufferu přesně tři bajty, jsou postupně přečteny a uloženy do paměti. V dalším kroku je nutné dekódovat kanál této \acs{MIDI} zprávy z \texttt{DATA BAJTU}~-- tuto \uv{extrakci} lze vyřešit pomocí bitového maskování. Takto získaný bajt je pak porovnáván s prvním nibblem bajtu \texttt{srcdstChannel} každého záznamu databáze \texttt{sub\-scrip\-tions}. Dojde-li ke shodě, začíná konstrukce odchozí zprávy podle dalších informací v odpovídajícím záznamu databáze spojení. V tomto bodě je důležité zmínit, že druhý nibble \texttt{DATA BAJTU}~-- kanál~-- je změněn, aby bylo zaručeno, že z \acs{MIDI} sběrnice cílového zařízení budou proudit zprávy na kanálech odpovídajících databázi spojení. Výpis kódu je k dispozici jako příloha \ref{code:handleMIDI}

\section{Přijetí UDP zprávy}
Díky architektuře odesílané zprávy (viz podkapitolu \ref{chpt:PrijMIDI}) je veškeré směrování dokončeno již na straně odesílatele. Přijímač může obdrženou zprávu tedy rovnou poslat do \acs{MIDI} výstupu.



%Nyní je procházena databáze připojení \texttt{sub\-scrip\-tions} a porovnáván první nibble bajtu \texttt{srcdstChannel} každého záznamu s 

\chapter{Průběh komunikace}
V~této kapitole bude skrz ukázkový příklad popsáno automatické rozpoznání přípravků na lokální síti, směrování \acs{MIDI} zpráv a konstrukce UDP datagramů.

\section{Připojení k lokální síti}
Každému přípravku byla do prvních šesti bajtů \texttt{EEPROM} paměti uložena unikátní MAC adresa. Při zapnutí a následné inicializaci je adresa načtena a přiřazena. Nejprve proběhne pokus o získání IP adresy pomocí DHCP serveru sítě. V případě neúspěchu použije přípravek uživatelem napevno nastavenou IP adresu. 

Díky knihovně \texttt{EthernetUdp.h} jsou tyto výše zmíněné operace otázkou pouze několika málo řádků:
\lstset{
    frame=single,
    keywordstyle=\color{blue},
    commentstyle=\color{olive}
    }
\begin{lstlisting}
if (Ethernet.begin())
{
    //DHCP server zdárně přidělil zařízení IP adresu
    Serial.println(Ethernet.localIP());
}
else
{
    Ethernet.begin(_myMac, _userIP);
    //Přidělení IP adresy napevno
    Serial.println(Ethernet.localIP());
}
\end{lstlisting}


\newcommand{\bytes}[4]{
    \begin{center}
        \large{\texttt{0x#1, 0x#2, 0x#3, 0x#4}}
    \end{center}
}



\section{Upozornění na vlastní přítomnost}
Ihned po zdárném připojení k lokální síti je vyslána zpráva  \texttt{beacon}, jejíž účel je upozornit všechna zařízení v lokální síti na vlastní přítomnost. Zpráva se skládá ze čtyř bajtů:
\bytes{FF}{FF}{FF}{FF}
a je poslána na broadcast adresu sítě.

\begin{lstlisting}
_broadcastIP = Ethernet.localIP();
_broadcastIP[3] = 255;
EthernetUdp eUDP;
eUDP.begin(MOE_PORT);
eUDP.beginPacket(_broadcastIP, MOE_PORT);
eUDP.write(_beacon, sizeof(_beacon));
eUDP.endPacket();
\end{lstlisting}




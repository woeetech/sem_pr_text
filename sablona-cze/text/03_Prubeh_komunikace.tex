\chapter{Průběh komunikace}
V~této kapitole bude skrz ukázkový příklad popsáno automatické rozpoznání přípravků na lokální síti, směrování \acs{MIDI} zpráv a konstrukce UDP datagramů.

\section{Připojení k lokální síti}
\subsection{Přiřazení MAC adresy a připojení k síti}
Každému přípravku byla do prvních šesti bajtů \texttt{EEPROM} paměti uložena unikátní MAC adresa. Při zapnutí a následné inicializaci je adresa načtena a přiřazena. Nejprve proběhne pokus o získání IP adresy pomocí DHCP serveru sítě. V případě neúspěchu použije přípravek uživatelem napevno nastavenou IP adresu. 

Díky knihovně \texttt{EthernetUdp.h} jsou tyto výše zmíněné operace otázkou pouze několika málo řádků:
\begin{lstlisting}[
    frame=single,
    keywordstyle=\color{blue},
    commentstyle=\color{olive}
]
if (Ethernet.begin())
{
    //DHCP server zdárně přidělil zařízení IP adresu
    Serial.println(Ethernet.localIP());
}
else
{
    Ethernet.begin(_myMac, _userIP);
    //Přidělení IP adresy napevno
    Serial.println(Ethernet.localIP());
}
\end{lstlisting}

\subsection{}
Ihned po zdárném připojení k lokální síti,


%\begin{lstlisting}[
%    frame=single,
%    numbers=left,
%    caption={},
%    label=lst:priklad.vypis.kodu.C,%basicstyle=\ttfamily\small, 
%    keywordstyle=\color{blue}
%    ]
%void Controller::macLoad(byte *mac)
%{
%    for (byte i = 0; i < 6; i++)
%    {
%        mac[i] = EEPROM.read(i);
%    }
%}
%\end{lstlisting}
